\usepackage{ifxetex,ifluatex}
\newif\ifxetexorluatex
\ifxetex
  \xetexorluatextrue
\else
  \ifluatex
    \xetexorluatextrue
  \else
    \xetexorluatexfalse
  \fi
\fi
\ifxetexorluatex
  \usepackage{fontspec}
\else
   \usepackage[T1]{fontenc}
\fi
\usepackage[normalem]{ulem}
\usepackage[utf8]{inputenc}
\usepackage{textcomp,lipsum,comment,tabularx,array}
\usepackage[section]{tocbibind}
\usepackage[version=3]{mhchem}
\includecomment{comment}
\usepackage[per-mode = symbol]{siunitx}
\DeclareSIUnit\curie{Ci}
\DeclareSIUnit\molar{M}
\setlength{\bibsep}{2pt}
\raggedright
\raggedbottom
\setlength{\parindent}{.3in}
\renewcommand{\nomname}{Abbreviations}
\def\nompreamble{\addcontentsline{toc}{section}{\nomname}\markboth{\nomname}{\nomname}}
\let\nomenclOrig\nomenclature
\renewcommand*{\nomenclature}[3][]{#2\nomenclOrig[#1]{#2}{#3}}
\renewcommand{\@makechapterhead}[1]{
 {\parindent \z@ \centering \normalfont
  \ifnum 
     \c@secnumdepth >\m@ne \bfseries \thechapter .
  \fi
  \bfseries \MakeUppercase{ #1 }\par
  \nobreak
  \vskip 20\p@
 }
}
\renewcommand\section{\@startsection {section}{1}{\z@}
  {-3.5ex \@plus -1ex \@minus -.2ex}
  {2.3ex \@plus.2ex}
  {\centering \normalfont \bfseries}}
